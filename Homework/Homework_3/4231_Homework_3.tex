\documentclass{../../class}

\title{4231 Homework 3}
\author{Eumin Hong (eh2890)}
\date{\today}

\begin{document}
\maketitle

\lhead{4231 Homework 3}
\rhead{Eumin Hong (eh2890)}

\tcbset{colback=blue!5, colframe=blue!75!black}

\section*{Problem 1}
\begin{tcolorbox}
    Prove the following useful property of a binary search tree (with distinct keys): \\

    \textbf{Property 1.} \textit{Let $x$ be a node in a BST $T$. Let $\max$ and $\min$ denote the largest and smallest keys in the subtree rooted at $x$, respectively. For any node $y$ outside the subtree rooted at $x$, show that either $y.key > \max$ or $y.key < \min$. This implies that if there is a key $k$ in the tree that satisfies $\min \leq k \leq \max$ then it must lie inside the subtree rooted at $x$. (Here the subtree rooted at $x$ includes $x$ itself.)} \\

    Use it to solve Exercise 12.2-5, 12.2-6, and 12.2-9 on page 293. In all three exercises, we assume the BST has distinct keys.
\end{tcolorbox}

\subsection*{12.2-5}
Show that if a node in a binary tree has two children, then its successor has no left child and its predecessor has no right child.

\subsection*{12.2-6}
Consider a binary search tree $T$ whose keys are distinct. Show that if the right subtree of a node $x$ in $T$ is empty and $x$ has a successor $y$, then $y$ is the lowest ancestor of $x$ whose left child is also an ancestor of $x$. (Recall that every node is its own ancestor.)

\subsection*{12.2-9}
Let $T$ be a binary search tree whose keys are distinct, let $x$ be a leaf node, and let $y$ be its parent. Show that $y.key$ is either the smallest key in $T$ larger than $x.key$ or the largest key in $T$ smaller than $x.key$.

\newpage
\section*{Problem 2}
\begin{tcolorbox}
    Problem 13-2 on page 332: Join operation. For a): you only need to answer the following two questions:
    \begin{enumerate}
        \item Let $T$ be a red-black tree in which the root has black height $T.bh$. Then after an insertion, $T.bh$ either stays the same or increases by $1$. Describe the scenario when it increases by $1$.
        \item If a node $z$ has black height $h$, use $O(1)$ time to compute the black height of $z$'s children.
    \end{enumerate} 
    Replace d) with the following: If $T_1.bh = T_2.bh$, what color should we make $x$ to get a red-black tree? If $T_1.bh > T_2.bh$, what color should we make $x$ so that properties 1, 2, 3, and 5 are maintained? \enquote{Briefly} describe how to enforce property 4 in $O(\lg{n})$ time.

    Skip e) and f).
\end{tcolorbox}

\subsection*{13-2}
The \textbf{\textit{join}} operation takes two dynamic sets $S_1$ and $S_2$ and an element $x$ such that for any $x_1 \in S_1$ and $x_2 \in S_2$, we have that $x_1.key \leq x.key \leq x_2.key$. It returns a set $S = S_1 \cup \{x\} \cup S_2$. In this problem, we investigate how to implement the join operation on red-black trees.
\begin{enumerate}[label=\textbf{\textit{\alph*}}.]
    \item From the original problem description:
    \begin{enumerate}[1.]
        \item Let $T$ be a red-black tree in which the root has black height $T.bh$. Then after an insertion, $T.bh$ either stays the same or increases by $1$. Describe the scenario when it increases by $1$.
        \item If a node $z$ has black height $h$, use $O(1)$ time to compute the black height of $z$'s children.
    \end{enumerate} 
\end{enumerate}

We wish to implement the operation $\textsc{RB-Join}(T_1, x, T_2)$, which destroys $T_1$ and $T_2$ and returns a red-black tree $T = T_1 \cup \{x\} \cup T_2$. Let $n$ be the total number of nodes in $T_1$ and $T_2$.
\begin{enumerate}[label=\textbf{\textit{\alph*}}.] \setcounter{enumi}{1}
    \item Assume that $T_1.bh \geq T_2.bh$. Describe an $O(\lg{n})$-time algorithm that finds a black node $y$ in $T_1$ with the largest key from among those nodes whose black-height is $T_2.bh$.
    \item Let $T_y$ be the subtree rooted at $y$. Describe how $T_y \cup \{x\} \cup T_2$ can replace $T_y$ in $O(1)$ time without destroying the binary-search-tree property.
    \item From the original problem description: If $T_1.bh = T_2.bh$, what color should we make $x$ so that properties 1, 2, 3, and 5 are maintained? \enquote{Briefly} describe how to enforce property 4 in $O(\lg{n})$ time.
\end{enumerate}

\newpage
\section*{Problem 3}
\begin{tcolorbox}
    Exercise 14.3-6 on page 354. You only need to describe the following key points:
    \begin{enumerate}
        \item What extra information to store in each node?
        \item With this additional information in each node, how to answer \textsc{Min-Gap} efficiently?
        \item Use Theorem $14.1$ to prove that insertion and deletion of a node can still be done in $O(\lg{n})$ time: Show that all the extra information for a node $x$ can be derived from the information stored in its two children in $O(1)$ time.
    \end{enumerate}
\end{tcolorbox}

\subsection*{14.3-6}
Show how to maintain a dynamic set $Q$ of numbers that supports the operation \textsc{Min-Gap}, which gives the magnitude of the difference of the two closest numbers in $Q$. For example, if $Q = \{1, 5, 9, 15, 18, 22\}$, then $\textsc{Min-Gap}(Q)$ returns $18-15=3$, since $15$ and $18$ are the two closest numers in $Q$. Make the operations \textsc{Insert}, \textsc{Delete}, \textsc{Search}, and \textsc{Min-Gap} as efficient as possible, and analyze their running times.

\newpage
\section*{Problem 4}
\begin{tcolorbox}
    Problem 16-1 (b) and (c) only on page 447.
\end{tcolorbox}
\subsection*{16-1}
Consider the problem of making change for $n$ cents using the fewest number of coins. Assume that each coin's value is an integer.
\begin{enumerate}[label=\textbf{\textit{\alph*}}.]
    \setcounter{enumi}{1}
    \item Suppose that the available coins are in the denominations that are powers of $c$, i.e., the denominations are $c^0, c^1, \dots, c^k$ for some integers $c > 1$ and $k \geq 1$. Show that the greedy algorithm always yields an optimal solution.
    \item Given a set of coin denominations for which the greedy algorithm does not yield an optimal solution. Your set should include a penny so that there is a solution for every value of $n$.
\end{enumerate}

\newpage
\section*{Problem 5}
\begin{tcolorbox}
    Problem 16-2 on page 447.
\end{tcolorbox}
Suppose you are given a set $S = \{a_1, a_2, \dots, a_n\}$ of tasks, where task $a_i$ requires $p_i$ units of processing time to complete, once it has started. You have one computer on which to these tasks, and the computer can run only one task at a time. Let $c_i$ be the \textbf{\textit{completion time}} of task $a_i$, that is, the time at which task $a_i$ completes processing. Your goal is to minimize the average completion time, that is, to minimize $(1 / n) \sum_{i=1}^{n} c_i$. For example, suppose there are two tasks, $a_1$ and $a_2$, with $p_1 = 3$ and $p_2 = 5$, and consider the schedule in which $a_2$ runs first, followed by $a_1$. Then $c_2 = 5, c_1 = 8$, and the average completion time is $(5 + 8) / 2 = 6.5$. If task $a_1$ runs first, however, then $c_1 = 3, c_2 = 8$, and the average completion time is $3 + 8) / 2 = 5.5$.
\begin{enumerate}[label=\textbf{\textit{\alph*}}]
    \item Give an algorithm that schedules the tasks so as to minimize the average completion time. Each task must run non-preemptively, that is, once task $a_i$ starts, it must run continuously for $p_i$ units of time. Prove that your algorithm minimizes the average completion time, and state the running time of your algorithm.
    \item Suppose now that the tasks are not all available at once. That is, each task cannot start until its \textbf{\textit{release time}} $r_i$. Suppose also that we allow \textbf{\textit{preemption}}, so that a task can be suspended and restarted at a later time. For example, a task $a_i$ with processing time $p_i = 6$ and release time $r_i = 1$ might start running at time $1$ and be preempted at time $4$. It might then resume at time $10$ but be preempted at time $11$, and it might finally resume at time $13$ and complete at time $15$. Task $a_i$ has run for a total of $6$ time units, but its running time has been divided into three pieces. In this scenario, $a_i$’s completion time is $15$. Give an algorithm that schedules the tasks so as to minimize the average completion time in this new scenario. Prove that your algorithm minimizes the average completion time, and state the running time of your algorithm.
\end{enumerate}

\end{document}
